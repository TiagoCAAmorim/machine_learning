\documentclass[final,5p]{elsarticle}

% \documentclass[preprint,12pt]{elsarticle}

%% Use the option review to obtain double line spacing
%% \documentclass[authoryear,preprint,review,12pt]{elsarticle}

%% Use the options 1p,twocolumn; 3p; 3p,twocolumn; 5p; or 5p,twocolumn
%% for a journal layout:
% \documentclass[final,1p,times]{elsarticle}
%% \documentclass[final,1p,times,twocolumn]{elsarticle}
% \documentclass[final,3p,times]{elsarticle}
%% \documentclass[final,3p,times,twocolumn]{elsarticle}
% \documentclass[final,5p,times]{elsarticle}
%% \documentclass[final,5p,times,twocolumn]{elsarticle}
\usepackage[portuguese]{babel}

%% For including figures, graphicx.sty has been loaded in
%% elsarticle.cls. If you prefer to use the old commands
%% please give \usepackage{epsfig}

%% The amssymb package provides various useful mathematical symbols
\usepackage{amssymb}
\usepackage{amsmath}
\usepackage{multirow}
\usepackage{tabularx}

\usepackage{pgfplots}
\pgfplotsset{compat=1.18}
\usepgfplotslibrary{statistics}
\usepackage{pgfplotstable}

\usepackage{placeins}
\usepackage{hyperref}
\numberwithin{equation}{section}

\usepackage{algorithm}
\usepackage[noEnd=true, indLines=true]{algpseudocodex}
\algrenewcommand\algorithmicrequire{\textbf{Entrada:}}
\algrenewcommand\algorithmicwhile{\textbf{Enquanto}}
\algrenewcommand\algorithmicrepeat{\textbf{Repete}}
\algrenewcommand\algorithmicuntil{\textbf{Até}}
\algrenewcommand\algorithmicif{\textbf{Se}}
\algrenewcommand\algorithmicthen{\textbf{então}}
\algrenewcommand\algorithmicelse{\textbf{Caso contrário}}
\algrenewcommand\algorithmicensure{\textbf{Objetivo:}}
\algrenewcommand\algorithmicreturn{\textbf{Retorna:}}
\algrenewcommand\algorithmicdo{\textbf{faça}}
\algrenewcommand\algorithmicforall{\textbf{Para todos}}
\algnewcommand{\LineComment}[1]{\State \(\triangleright\) \textcolor{black!50}{\emph{#1}}}

\newcommand*{\squareb}{\textcolor{black}{\rule{0.5em}{0.5em}}}
\newcommand*{\squareg}{\textcolor{gray}{\rule{0.5em}{0.5em}}}

% \usepackage[fleqn]{nccmath}
% \usepackage{multicol}


%=========== Gloabal Tikz settings
% \pgfplotsset{compat=newest}
% \usetikzlibrary{math}
% \pgfplotsset{
%     height = 10cm,
%     width = 10cm,
%     tick pos = left,
%     legend style={at={(0.98,0.30)}, anchor=east},
%     legend cell align=left,
%     }
%  \pgfkeys{
%     /pgf/number format/.cd,
%     fixed,
%     precision = 1,
%     set thousands separator = {}
% }

%% The amsthm package provides extended theorem environments
%% \usepackage{amsthm}

%% The lineno packages adds line numbers. Start line numbering with
%% \begin{linenumbers}, end it with \end{linenumbers}. Or switch it on
%% for the whole article with \linenumbers.
%% \usepackage{lineno}

\usepackage{listings}
\usepackage{xcolor}

\definecolor{codegreen}{rgb}{0,0.6,0}
\definecolor{codegray}{rgb}{0.5,0.5,0.5}
\definecolor{codepurple}{rgb}{0.58,0,0.82}
\definecolor{backcolour}{rgb}{0.98,0.98,0.98}

\lstdefinestyle{mystyle}{
    backgroundcolor=\color{backcolour},
    commentstyle=\color{codegreen},
    keywordstyle=\color{magenta},
    numberstyle=\tiny\color{codegray},
    stringstyle=\color{codepurple},
    basicstyle=\ttfamily\footnotesize,
    breakatwhitespace=false,
    breaklines=true,
    captionpos=b,
    keepspaces=true,
    numbers=left,
    numbersep=5pt,
    showspaces=false,
    showstringspaces=false,
    showtabs=false,
    tabsize=2
}

\lstset{style=mystyle}

% \journal{Nuclear Physics B}

\begin{document}

\begin{frontmatter}

%% Title, authors and addresses

%% use the tnoteref command within \title for footnotes;
%% use the tnotetext command for theassociated footnote;
%% use the fnref command within \author or \address for footnotes;
%% use the fntext command for theassociated footnote;
%% use the corref command within \author for corresponding author footnotes;
%% use the cortext command for theassociated footnote;
%% use the ead command for the email address,
%% and the form \ead[url] for the home page:
%% \title{Title\tnoteref{label1}}
%% \tnotetext[label1]{}
%% \author{Name\corref{cor1}\fnref{label2}}
%% \ead{email address}
%% \ead[url]{home page}
%% \fntext[label2]{}
%% \cortext[cor1]{}
%% \affiliation{organization={},
%%             addressline={},
%%             city={},
%%             postcode={},
%%             state={},
%%             country={}}
%% \fntext[label3]{}

\title{Aplicação de Regressão Linear a Série Temporal\tnoteref{label_title}}
\tnotetext[label_title]{Relatório número 01 como parte dos requisitos da disciplina IA048: Aprendizado de Máquina.}

%% use optional labels to link authors explicitly to addresses:
%% \author[label1,label2]{}
%% \affiliation[label1]{organization={},
%%             addressline={},
%%             city={},
%%             postcode={},
%%             state={},
%%             country={}}
%%
%% \affiliation[label2]{organization={},
%%             addressline={},
%%             city={},
%%             postcode={},
%%             state={},
%%             country={}}

\author[label1]{Tiago C A Amorim (RA: 100675)}
\affiliation[label1]{organization={Departamento de Engenharia de Petróleo da Faculdade de Engenharia Mecânica, UNICAMP},
            city={Campinas},
            state={SP},
            country={Brasil}}

\author[label2]{Taylon Xxxxxxx (RA: xxxxxx)}
\affiliation[label2]{organization={Departamento XXXX, UNICAMP},
            city={Campinas},
            state={SP},
            country={Brasil}}


\begin{abstract}

    xxxxxxx

\end{abstract}


%%Graphical abstract
% \begin{graphicalabstract}
%\includegraphics{grabs}
% \end{graphicalabstract}

%%Research highlights
% \begin{highlights}
% \item Research highlight 1
% \item Research highlight 2
% \end{highlights}

\begin{keyword}
    Regressão Linear \sep Séries Temporais \sep Validação Cruzada
%% keywords here, in the form: keyword \sep keyword

%% PACS codes here, in the form: \PACS code \sep code

%% MSC codes here, in the form: \MSC code \sep code
%% or \MSC[2008] code \sep code (2000 is the default)

\end{keyword}

\end{frontmatter}

%% main text
\section{Introdução}


\section{Tarefa Proposta}

    Trabalhar com a base de dados U.S. Airline Traffic Data, a qual contém informações referentes ao tráfego aéreo mensal norte-americano no período de 2003 a 2023, disponibilizadas pelo \emph{U.S. Department of Transportation’s (DOT) Bureau of Transportation Statistics}. Em particular, vamos explorar a série temporal do número total de vôos (domésticos e internacionais).

    Explorar um modelo linear para a previsão considerando que o horizonte de predição é L = 1 (passos à frente da série temporal).

    \begin{enumerate}[(a)]
        \item Exiba o gráfico da série temporal completa. Numa inspeção visual simples, é possível reconhecer ao menos três faixas distintas de comportamento aproximadamente “regular” na série:

        \begin{enumerate}[(i)]
            \item Jan/2003 a Ago/2008.
            \item Set/2008 a Dez/2019.
            \item Jan/2020 a Set/2023.
        \end{enumerate}

        Discuta possíveis razões históricas/econômicas para as transições de comportamento.

        \item Divida a série em dois conjuntos:

        \begin{enumerate}[(i)]
            \item \textbf{treinamento} e \textbf{validação}: com amostras de 2003 a 2019.
            \item \textbf{teste}: com amostras de 2020 a 2023.
        \end{enumerate}

        Faça a análise de desempenho do preditor linear ótimo, no sentido de quadrados mínimos irrestrito considerando:

        \begin{enumerate}[(1)]
            \item A progressão do valor da raiz quadrada do erro quadrático médio (RMSE, do inglês \emph{root mean squared error}), junto aos dados de validação, em função do número de entradas (K) do preditor (desde K = 1 a K = 24). Apresente o gráfico obtido e busque tecer conjecturas sobre os motivos subjacentes a seu comportamento. \label{item:b1}
            \item O gráfico com as amostras de teste da série temporal e as respetivas estimativas geradas pela melhor versão do preditor (i.e., usando o valor de K que levou ao mínimo erro de validação). Obtenha, também, o RMSE e o erro percentual absoluto médio (MAPE, do inglês \emph{mean absolute percentage error}) para o conjunto de teste.  \label{item:b2}
            \item O gráfico com as amostras apenas dos dois últimos anos (2022 e 2023) e as estimativas geradas pelo melhor preditor, além dos respetivos valores de RMSE e MAPE.  \label{item:b3}
        \end{enumerate}

        \item Repita o procedimento detalhado nos itens \ref{item:b1} e \ref{item:b2}, mas adotando a seguinte divisão dos dados:

        \begin{enumerate}[(i)]
            \item \textbf{treinamento}: amostras de 2003 a 2019.
            \item \textbf{validação}: amostras de 2020 e 2021.
            \item \textbf{teste}: amostras de 2022 e 2023.
        \end{enumerate}

        Discuta os resultados obtidos e faça uma comparação com o cenário anterior (especialmente
        com o que foi obtido no item \ref{item:b3}).
    \end{enumerate}

\section{Aplicação}

    Toda a avaliação foi feita em um único \emph{notebook} Jupyter, em Python. Foi feito o uso da biblioteca \emph{SciKit-Learn} \cite{scikit-learn} para realizar as diferentes manipulações nos dados.

    \subsection{Avaliação do Conjunto de Dados}

    O conjunto de dados utilizado nesta avaliação é o de tráfego aéreo dos EUA disponibilizado no Kaggle \cite{YYXian_2024}. Segundo o autor este conjunto de dados fornece o tráfego aéreo mensal dos EUA de 2003 a 2023, incluindo o número de passageiros, o número de voos, as milhas de passageiros pagantes, as milhas de assentos disponíveis e o fator de ocupação.

    Nesta avaliação o foco é no número total de vôos.







    \subsection{Problema proposto}


    \section{Implementação}


\section{Resultados}


        O código foi implementado em Python em um \emph{notebook} Jupyter. Pode ser encontrado em \href{https://github.com/TiagoCAAmorim/machine_learning/blob/main/Lista01/Lista01.ipynb}{https://github.com/Tiago CAAmorim/machine\underline{ }learning}.

\section{Conclusão}


    % \label{}

%% The Appendices part is started with the command \appendix;
%% appendix sections are then done as normal sections

\appendix

\section{Lista de Variáveis}

    % \begin{description}
    %     \item[$Bo$:]Fator volume de formação do óleo no reservatório ($m^3/m^3$).
    %     \item[$Bw$:]Fator volume de formação da água no reservatório ($m^3/m^3$).
    %     \item[$D$:]Profundidade.
    %     \item[$\Delta x$:]Discretização espacial na direção $i$.
    %     \item[$\Delta y$:]Discretização espacial na direção $j$.
    %     \item[$\Delta z$:]Discretização espacial na direção $k$.
    %     \item[$\Delta t$:]Discretização temporal.
    %     \item[$\gamma_p$:]Peso específico da fase $p$ ($\gamma_p = \rho_p g$).
    %     \item[$k$:]Permeabilidade absoluta do meio poroso.
    %     \item[$k_{rp}$:]Permeabilidade relativa da fase $p$.
    %     \item[$\mu_p$:]Viscosidade da fase $p$.
    %     \item[$n_i$:]Número de células na direção $i$.
    %     \item[$n_j$:]Número de células na direção $j$.
    %     \item[$p$:]Pressão.
    %     \item[$p_{wf}$:]Pressão de fundo do poço.
    %     \item[$\phi$:]Porosidade da rocha.
    %     \item[$q_p$:]Vazão volumétrica da fase $p$.
    %     \item[$q^{std}_{p}$:]Vazão volumétrica da fase $p$ medida em condições padrão (\emph{standard}).
    %     \item[$\rho_p$:]Densidade da fase $p$.
    %     \item[$r_w$:]Raio do poço.
    %     \item[$S_p$:]Saturação da fase $p$ no meio poroso.
    %     \item[$Swi$:]Saturação de água inicial (imóvel).
    %     \item[$Sor$:]Saturação de óleo residual (imóvel).
    % \end{description}


%% \section{}
%% \label{}

%% If you have bibdatabase file and want bibtex to generate the
%% bibitems, please use
%%

\bibliographystyle{elsarticle-num}
\bibliography{refs}

%% else use the following coding to input the bibitems directly in the
%% TeX file.

% \begin{thebibliography}{00}

%% \bibitem{label}
%% Text of bibliographic item

% \bibitem{}

% \end{thebibliography}

% \newpage
% \FloatBarrier
% \section{Código em C}

% O código de ambos métodos foi implementado em um único arquivo. O código é apresentado em duas partes neste documento para facilitar a leitura. O código pode ser encontrado em \href{https://github.com/TiagoCAAmorim/numerical-methods}{https://github.com/TiagoCAAmorim/numerical-methods}.

% \subsection{Método da Bissecção}
% \lstinputlisting[language=C, linerange={1-229}]{./02_newton_raphson.c}

% \subsection{Método de Newton-Raphson}
% \lstinputlisting[language=C, linerange={231-445}]{./02_newton_raphson.c}

% \subsection{Método da Mínima Curvatura}
% \lstinputlisting[language=C, linerange={448-958}]{./02_newton_raphson.c}

\end{document}
\endinput