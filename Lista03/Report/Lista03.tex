\documentclass[final,5p]{elsarticle}

% \documentclass[preprint,12pt]{elsarticle}

%% Use the option review to obtain double line spacing
%% \documentclass[authoryear,preprint,review,12pt]{elsarticle}

%% Use the options 1p,twocolumn; 3p; 3p,twocolumn; 5p; or 5p,twocolumn
%% for a journal layout:
% \documentclass[final,1p,times]{elsarticle}
%% \documentclass[final,1p,times,twocolumn]{elsarticle}
% \documentclass[final,3p,times]{elsarticle}
%% \documentclass[final,3p,times,twocolumn]{elsarticle}
% \documentclass[final,5p,times]{elsarticle}
%% \documentclass[final,5p,times,twocolumn]{elsarticle}
\usepackage[portuguese]{babel}

%% For including figures, graphicx.sty has been loaded in
%% elsarticle.cls. If you prefer to use the old commands
%% please give \usepackage{epsfig}

%% The amssymb package provides various useful mathematical symbols
\usepackage{amssymb}
\usepackage{amsmath}
\usepackage{multirow}
\usepackage{tabularx}
\usepackage{booktabs}
\usepackage{tablefootnote}

\usepackage{pgfplots}
\pgfplotsset{compat=1.18}
\usepgfplotslibrary{statistics}
\usepackage{pgfplotstable}

\usepackage{placeins}
\usepackage{hyperref}
\numberwithin{equation}{section}

\usepackage{algorithm}
\usepackage[noEnd=true, indLines=true]{algpseudocodex}
\algrenewcommand\algorithmicrequire{\textbf{Entrada:}}
\algrenewcommand\algorithmicwhile{\textbf{Enquanto}}
\algrenewcommand\algorithmicrepeat{\textbf{Repete}}
\algrenewcommand\algorithmicuntil{\textbf{Até}}
\algrenewcommand\algorithmicif{\textbf{Se}}
\algrenewcommand\algorithmicthen{\textbf{então}}
\algrenewcommand\algorithmicelse{\textbf{Caso contrário}}
\algrenewcommand\algorithmicensure{\textbf{Objetivo:}}
\algrenewcommand\algorithmicreturn{\textbf{Retorna:}}
\algrenewcommand\algorithmicdo{\textbf{faça}}
\algrenewcommand\algorithmicforall{\textbf{Para todos}}
\algnewcommand{\LineComment}[1]{\State \(\triangleright\) \textcolor{black!50}{\emph{#1}}}

\newcommand*{\squareb}{\textcolor{black}{\rule{0.5em}{0.5em}}}
\newcommand*{\squareg}{\textcolor{gray}{\rule{0.5em}{0.5em}}}

\graphicspath{ {./png/} }

% \usepackage[fleqn]{nccmath}
% \usepackage{multicol}


%=========== Gloabal Tikz settings
% \pgfplotsset{compat=newest}
% \usetikzlibrary{math}
% \pgfplotsset{
%     height = 10cm,
%     width = 10cm,
%     tick pos = left,
%     legend style={at={(0.98,0.30)}, anchor=east},
%     legend cell align=left,
%     }
%  \pgfkeys{
%     /pgf/number format/.cd,
%     fixed,
%     precision = 1,
%     set thousands separator = {}
% }

%% The amsthm package provides extended theorem environments
%% \usepackage{amsthm}

%% The lineno packages adds line numbers. Start line numbering with
%% \begin{linenumbers}, end it with \end{linenumbers}. Or switch it on
%% for the whole article with \linenumbers.
%% \usepackage{lineno}

\usepackage{listings}
\usepackage{xcolor}

\definecolor{codegreen}{rgb}{0,0.6,0}
\definecolor{codegray}{rgb}{0.5,0.5,0.5}
\definecolor{codepurple}{rgb}{0.58,0,0.82}
\definecolor{backcolour}{rgb}{0.98,0.98,0.98}

\lstdefinestyle{mystyle}{
    backgroundcolor=\color{backcolour},
    commentstyle=\color{codegreen},
    keywordstyle=\color{magenta},
    numberstyle=\tiny\color{codegray},
    stringstyle=\color{codepurple},
    basicstyle=\ttfamily\footnotesize,
    breakatwhitespace=false,
    breaklines=true,
    captionpos=b,
    keepspaces=true,
    numbers=left,
    numbersep=5pt,
    showspaces=false,
    showstringspaces=false,
    showtabs=false,
    tabsize=2
}

\lstset{style=mystyle}

% \journal{Nuclear Physics B}

\begin{document}

\begin{frontmatter}

%% Title, authors and addresses

%% use the tnoteref command within \title for footnotes;
%% use the tnotetext command for theassociated footnote;
%% use the fnref command within \author or \address for footnotes;
%% use the fntext command for theassociated footnote;
%% use the corref command within \author for corresponding author footnotes;
%% use the cortext command for theassociated footnote;
%% use the ead command for the email address,
%% and the form \ead[url] for the home page:
%% \title{Title\tnoteref{label1}}
%% \tnotetext[label1]{}
%% \author{Name\corref{cor1}\fnref{label2}}
%% \ead{email address}
%% \ead[url]{home page}
%% \fntext[label2]{}
%% \cortext[cor1]{}
%% \affiliation{organization={},
%%             addressline={},
%%             city={},
%%             postcode={},
%%             state={},
%%             country={}}
%% \fntext[label3]{}

\title{Classificação de Imagens com Redes Neurais Artificiais\tnoteref{label_title}}
\tnotetext[label_title]{Relatório número 03 como parte dos requisitos da disciplina IA048: Aprendizado de Máquina.}

%% use optional labels to link authors explicitly to addresses:
%% \author[label1,label2]{}
%% \affiliation[label1]{organization={},
%%             addressline={},
%%             city={},
%%             postcode={},
%%             state={},
%%             country={}}
%%
%% \affiliation[label2]{organization={},
%%             addressline={},
%%             city={},
%%             postcode={},
%%             state={},
%%             country={}}

\author[label1]{Tiago C A Amorim (RA: 100675)}
\affiliation[label1]{organization={Doutorando no Departamento de Engenharia de Petróleo da Faculdade de Engenharia Mecânica, UNICAMP},
            city={Campinas},
            state={SP},
            country={Brasil}}

\author[label2]{Taylon L C Martins (RA: 177379)}
\affiliation[label2]{organization={Aluno especial, UNICAMP},
            city={Campinas},
            state={SP},
            country={Brasil}}


% \begin{abstract}

%     xxxxxxx

% \end{abstract}


%%Graphical abstract
% \begin{graphicalabstract}
%\includegraphics{grabs}
% \end{graphicalabstract}

%%Research highlights
% \begin{highlights}
% \item Research highlight 1
% \item Research highlight 2
% \end{highlights}

\begin{keyword}
    Classificação \sep Redes Neurais Artificiais
%% keywords here, in the form: keyword \sep keyword

%% PACS codes here, in the form: \PACS code \sep code

%% MSC codes here, in the form: \MSC code \sep code
%% or \MSC[2008] code \sep code (2000 is the default)

\end{keyword}

\end{frontmatter}

%% main text
\section{Introdução}

    Este relatório apresenta as principais atividades realizadas no desenvolvimento das atividades propostas na Lista 03 da disciplina IA048: Aprendizado de Máquina, primeiro semestre de 2024. O foco deste exercício é de construir a avaliar o desempenho de redes neurais artificiais, MLP (densas de uma camada) e CNN (convolucionais), na classificação de imagens de células sanguíneas periféricas.

\section{Tarefa Proposta}

    Nesta atividade, vamos abordar o problema de reconhecimento de células sanguíneas periféricas utilizando a base de dados BloodMNIST \cite{acevedo2020dataset,yang2110v2} (\href{https://medmnist.com/}{https://medmnist.com/}), a qual possui 17.092 imagens microscópicas coloridas (3 canais de cor). O mapeamento entre os identificadores das classes e os rótulos está indicado na Tabela \ref{tab:rotulos}.

    \begin{table}[h]
        \centering
        \begin{tabular}{l l c}
            \toprule
            \textbf{Id} & \textbf{Rótulo} \\
            \midrule
            0 & Basófilos \\
            1 & Eosinófilos \\
            2 & Eritroblastos \\
            3 & Granulócitos imaturos \\
            4 & Linfócitos \\
            5 & Monócitos \\
            6 & Neutrófilos \\
            7 & Plaquetas \\
            \bottomrule
        \end{tabular}
        \caption{Correspondência entre os identificadores numéricos das classes e os tipos de células sanguíneas.}
        \label{tab:rotulos}
    \end{table}

    \begin{enumerate}[(a)]
        \item Aplique uma rede MLP com uma camada intermediária e analise (1) a acurácia e (2) a matriz de confusão para os dados de teste obtidas pela melhor versão desta rede. Descreva a metodologia e a arquitetura empregada, bem como todas as escolhas feitas.
        \item Monte uma CNN simples contendo:
        \begin{itemize}
            \item Uma camada convolucional com função de ativação não-linear.
            \item Uma camada de \emph{pooling}.
            \item Uma camada de saída do tipo \emph{softmax}.
        \end{itemize}
        Avalie a progressão da acurácia junto aos dados de validação em função:
        \begin{itemize}
            \item Da quantidade de \emph{kernels} utilizados na camada convolucional;
            \item Do tamanho do \emph{kernel} de convolução.
        \end{itemize}
        \item Escolhendo, então, a melhor configuração para a CNN simples, refaça o treinamento do modelo e apresente:
        \begin{itemize}
            \item A matriz de confusão para os dados de teste;
            \item A acurácia global;
            \item Cinco padrões de teste que foram classificados incorretamente, indicando a classe esperada e as probabilidades estimadas pela rede.
        \end{itemize}
        Discuta os resultados obtidos.
        \item Explore, agora, uma CNN um pouco mais profunda. Descreva a arquitetura utilizada e apresente os mesmos resultados solicitados no item (c) para o conjunto de teste. Por fim, faça uma breve comparação entre os modelos estudados neste exercício.
    \end{enumerate}


\section{Aplicação}

    Toda a avaliação foi feita em um único \emph{notebook} Jupyter, em Python. Foi feito o uso da biblioteca \emph{TensorFlow} \cite{tensorflow2015-whitepaper} para fazer as diferentes manipulações nos dados. O código pode ser encontrado em \href{https://github.com/TiagoCAAmorim/machine_learning/blob/main/Lista03/Lista03.ipynb}{https://github.com/Tiago CAAmorim/machine\_learning}.

    \subsection{Rede MLP}

    \subsection{Rede Convolucional \emph{Simples}}

    \subsection{Rede Convolucional \emph{Profunda}}

\section{Análise dos Resultados}


\section{Conclusão}


% \appendix

% \section{Lista de Variáveis}

    % \begin{description}
    %     \item[$X$:] xxxxxxxxxxx.
    % \end{description}


%% \section{}
%% \label{}

%% If you have bibdatabase file and want bibtex to generate the
%% bibitems, please use
%%

\bibliographystyle{elsarticle-num}
\bibliography{refs}

%% else use the following coding to input the bibitems directly in the
%% TeX file.

% \begin{thebibliography}{00}

%% \bibitem{label}
%% Text of bibliographic item

% \bibitem{}

% \end{thebibliography}


\end{document}
\endinput
