\documentclass[final,5p]{elsarticle}

% \documentclass[preprint,12pt]{elsarticle}

%% Use the option review to obtain double line spacing
%% \documentclass[authoryear,preprint,review,12pt]{elsarticle}

%% Use the options 1p,twocolumn; 3p; 3p,twocolumn; 5p; or 5p,twocolumn
%% for a journal layout:
% \documentclass[final,1p,times]{elsarticle}
%% \documentclass[final,1p,times,twocolumn]{elsarticle}
% \documentclass[final,3p,times]{elsarticle}
%% \documentclass[final,3p,times,twocolumn]{elsarticle}
% \documentclass[final,5p,times]{elsarticle}
%% \documentclass[final,5p,times,twocolumn]{elsarticle}
\usepackage[portuguese]{babel}

%% For including figures, graphicx.sty has been loaded in
%% elsarticle.cls. If you prefer to use the old commands
%% please give \usepackage{epsfig}

%% The amssymb package provides various useful mathematical symbols
\usepackage{amssymb}
\usepackage{amsmath}
\usepackage{multirow}
\usepackage{tabularx}

\usepackage{pgfplots}
\pgfplotsset{compat=1.18}
\usepgfplotslibrary{statistics}
\usepackage{pgfplotstable}

\usepackage{placeins}
\usepackage{hyperref}
\numberwithin{equation}{section}

\usepackage{algorithm}
\usepackage[noEnd=true, indLines=true]{algpseudocodex}
\algrenewcommand\algorithmicrequire{\textbf{Entrada:}}
\algrenewcommand\algorithmicwhile{\textbf{Enquanto}}
\algrenewcommand\algorithmicrepeat{\textbf{Repete}}
\algrenewcommand\algorithmicuntil{\textbf{Até}}
\algrenewcommand\algorithmicif{\textbf{Se}}
\algrenewcommand\algorithmicthen{\textbf{então}}
\algrenewcommand\algorithmicelse{\textbf{Caso contrário}}
\algrenewcommand\algorithmicensure{\textbf{Objetivo:}}
\algrenewcommand\algorithmicreturn{\textbf{Retorna:}}
\algrenewcommand\algorithmicdo{\textbf{faça}}
\algrenewcommand\algorithmicforall{\textbf{Para todos}}
\algnewcommand{\LineComment}[1]{\State \(\triangleright\) \textcolor{black!50}{\emph{#1}}}

\newcommand*{\squareb}{\textcolor{black}{\rule{0.5em}{0.5em}}}
\newcommand*{\squareg}{\textcolor{gray}{\rule{0.5em}{0.5em}}}

\graphicspath{ {./png/} }

% \usepackage[fleqn]{nccmath}
% \usepackage{multicol}


%=========== Gloabal Tikz settings
% \pgfplotsset{compat=newest}
% \usetikzlibrary{math}
% \pgfplotsset{
%     height = 10cm,
%     width = 10cm,
%     tick pos = left,
%     legend style={at={(0.98,0.30)}, anchor=east},
%     legend cell align=left,
%     }
%  \pgfkeys{
%     /pgf/number format/.cd,
%     fixed,
%     precision = 1,
%     set thousands separator = {}
% }

%% The amsthm package provides extended theorem environments
%% \usepackage{amsthm}

%% The lineno packages adds line numbers. Start line numbering with
%% \begin{linenumbers}, end it with \end{linenumbers}. Or switch it on
%% for the whole article with \linenumbers.
%% \usepackage{lineno}

\usepackage{listings}
\usepackage{xcolor}

\definecolor{codegreen}{rgb}{0,0.6,0}
\definecolor{codegray}{rgb}{0.5,0.5,0.5}
\definecolor{codepurple}{rgb}{0.58,0,0.82}
\definecolor{backcolour}{rgb}{0.98,0.98,0.98}

\lstdefinestyle{mystyle}{
    backgroundcolor=\color{backcolour},
    commentstyle=\color{codegreen},
    keywordstyle=\color{magenta},
    numberstyle=\tiny\color{codegray},
    stringstyle=\color{codepurple},
    basicstyle=\ttfamily\footnotesize,
    breakatwhitespace=false,
    breaklines=true,
    captionpos=b,
    keepspaces=true,
    numbers=left,
    numbersep=5pt,
    showspaces=false,
    showstringspaces=false,
    showtabs=false,
    tabsize=2
}

\lstset{style=mystyle}

% \journal{Nuclear Physics B}

\begin{document}

\begin{frontmatter}

%% Title, authors and addresses

%% use the tnoteref command within \title for footnotes;
%% use the tnotetext command for theassociated footnote;
%% use the fnref command within \author or \address for footnotes;
%% use the fntext command for theassociated footnote;
%% use the corref command within \author for corresponding author footnotes;
%% use the cortext command for theassociated footnote;
%% use the ead command for the email address,
%% and the form \ead[url] for the home page:
%% \title{Title\tnoteref{label1}}
%% \tnotetext[label1]{}
%% \author{Name\corref{cor1}\fnref{label2}}
%% \ead{email address}
%% \ead[url]{home page}
%% \fntext[label2]{}
%% \cortext[cor1]{}
%% \affiliation{organization={},
%%             addressline={},
%%             city={},
%%             postcode={},
%%             state={},
%%             country={}}
%% \fntext[label3]{}

\title{Avaliação de Técnicas de Classificação\tnoteref{label_title}}
\tnotetext[label_title]{Relatório número 02 como parte dos requisitos da disciplina IA048: Aprendizado de Máquina.}

%% use optional labels to link authors explicitly to addresses:
%% \author[label1,label2]{}
%% \affiliation[label1]{organization={},
%%             addressline={},
%%             city={},
%%             postcode={},
%%             state={},
%%             country={}}
%%
%% \affiliation[label2]{organization={},
%%             addressline={},
%%             city={},
%%             postcode={},
%%             state={},
%%             country={}}

\author[label1]{Tiago C A Amorim (RA: 100675)}
\affiliation[label1]{organization={Doutorando no Departamento de Engenharia de Petróleo da Faculdade de Engenharia Mecânica, UNICAMP},
            city={Campinas},
            state={SP},
            country={Brasil}}

\author[label2]{Taylon L C Martins (RA: 177379)}
\affiliation[label2]{organization={Aluno especial, UNICAMP},
            city={Campinas},
            state={SP},
            country={Brasil}}


% \begin{abstract}

%     xxxxxxx

% \end{abstract}


%%Graphical abstract
% \begin{graphicalabstract}
%\includegraphics{grabs}
% \end{graphicalabstract}

%%Research highlights
% \begin{highlights}
% \item Research highlight 1
% \item Research highlight 2
% \end{highlights}

\begin{keyword}
    Classificação \sep Regressão Logística \sep k-Vizinhos mais Próximos \sep Validação Cruzada
%% keywords here, in the form: keyword \sep keyword

%% PACS codes here, in the form: \PACS code \sep code

%% MSC codes here, in the form: \MSC code \sep code
%% or \MSC[2008] code \sep code (2000 is the default)

\end{keyword}

\end{frontmatter}

%% main text
\section{Introdução}

    Este relatório apresenta as principais atividades realizadas no desenvolvimento das atividades propostas na Lista 02 da disciplina IA048: Aprendizado de Máquina, primeiro semestre de 2024. O foco deste exercício é de construir a avaliar o desempenho de algoritmos de classificação.

\section{Tarefa Proposta}

    Nesta atividade, vamos abordar o problema de reconhecimento de atividades humanas (HAR, do inglês \emph{human activity recognition}) a partir de informações capturadas por sensores de smartphones. Em particular, vamos trabalhar com a base de dados \href{https://archive.ics.uci.edu/dataset/240/human+activity+recognition+using+smartphones}{UCI HAR} \cite{anguita2013public}, que contém registros de sensores inerciais presentes em um smartphone preso à cintura de 30 sujeitos realizando atividades cotidianas. Cada pessoa realizou seis atividades, as quais correspondem aos seguintes rótulos:

    \begin{table}[h]
        \centering
        \begin{tabular}{c c}
            \hline
            Atividade & Rótulo \\
            \hline
            Caminhar & 0 \\
            Subir escadas & 1 \\
            Descer escadas & 2 \\
            Sentado & 3 \\
            Em pé & 4 \\
            Deitado & 5 \\
            \hline
        \end{tabular}
        \caption{Rótulos}
        \label{tab:rotulos}
    \end{table}

    Foram capturadas as amostras dos três eixos (x, y e z) do acelerômetro (ACC, do inglês \emph{accelerometer}) e do giroscópio (GYR, do inglês \emph{gyroscope}) presentes no smartphone, empregando uma taxa de amostragem de 50 Hz. O conjunto completo de amostras foi particionado aleatoriamente em treinamento (70\% dos voluntários) e teste (30\% dos voluntários).

    \subsection{Primeira parte}

        Primeiramente, será explorada uma versão do conjunto de dados na qual já houve pré-processamento e extração de características. No caso, cada amostra contém 561 atributos derivados de uma mesma janela de 2,56 s dos 6 sinais disponíveis (ACC: x,y,z; GYR: x,y,z), considerando suas representações tanto no domínio do tempo quanto no domínio da frequência.

        \begin{enumerate}[(a)]
            \item Construa uma solução para este problema baseada no modelo de regressão logística. Descreva a abordagem escolhida para resolvê-lo (softmax, classificadores binários combinados em um esquema um-contra-um ou um-contra-todos). Obtenha, então, a matriz de confusão para o classificador considerando os dados do conjunto de teste. Além disso, adote uma métrica global para a avaliação do desempenho (médio) deste classificador. Discuta os resultados obtidos.
            \item Considere, agora, a técnica k-nearest neighbors (kNN). Adotando um esquema de validação cruzada, mostre como o desempenho do classificador, computado com a mesma métrica adotada no item (a) varia em função do parâmetro k. Escolhendo, então, o melhor valor para k, apresente a matriz de confusão para os dados de teste e o desempenho medido nesse conjunto. Comente os resultados obtidos, inclusive estabelecendo uma comparação com o desempenho da regressão logística.
        \end{enumerate}

    \subsection{Segunda parte}

        Agora, vamos utilizar os dados “brutos” combinados de ACC e GYR como entradas dos classificadores. Para isso, devemos recorrer aos registros disponibilizados no diretório 'Inertial Signals', os quais estão separados por eixo e por sensor, sendo que cada amostra individual agora é formada por 128 valores (atributos), que correspondem às amplitudes instantâneas de aceleração (ACC) ou velocidade angular (GYR) dentro de uma janela de 2,56 s.

        \setcounter{enumi}{3}
        \begin{enumerate}[(a)]
            \item Monte, então, a nova matriz de entrada concatenando os seis sinais temporais e, então, repita o procedimento experimental detalhado nos itens (a) e (b). Ao final, com base no desempenho obtido, teça uma análise comparativa entre a abordagem do item anterior e a abordagem baseada nos sinais “brutos” empregada nesta segunda parte.
        \end{enumerate}




\section{Aplicação}

    Toda a avaliação foi feita em um único \emph{notebook} Jupyter, em Python. Foi feito o uso da biblioteca \emph{Scikit-learn} \cite{scikit-learn} para fazer as diferentes manipulações nos dados. O código pode ser encontrado em \href{https://github.com/TiagoCAAmorim/machine_learning/blob/main/Lista02/Lista02.ipynb}{https://github.com/Tiago CAAmorim/machine\_learning}.

    \subsection{Avaliação do Conjunto de Dados}

        \subsubsection{Pré-processamento}

    \subsection{Primeiro Modelo de Classificação}

        \subsubsection{Normalização dos Dados}

        \subsubsection{Busca pelo Melhor Modelo}

        \subsubsection{Erros com os Dados de Teste} \label{sec:testes_modelo}


    \subsection{Segundo Modelo de Regressão}


\section{Conclusão}


\appendix

% \section{Lista de Variáveis}

    % \begin{description}
    %     \item[$X$:] xxxxxxxxxxx.
    % \end{description}


%% \section{}
%% \label{}

%% If you have bibdatabase file and want bibtex to generate the
%% bibitems, please use
%%

\bibliographystyle{elsarticle-num}
\bibliography{refs}

%% else use the following coding to input the bibitems directly in the
%% TeX file.

% \begin{thebibliography}{00}

%% \bibitem{label}
%% Text of bibliographic item

% \bibitem{}

% \end{thebibliography}

% \newpage
% \FloatBarrier
% \section{Código em C}

% O código de ambos métodos foi implementado em um único arquivo. O código é apresentado em duas partes neste documento para facilitar a leitura. O código pode ser encontrado em \href{https://github.com/TiagoCAAmorim/numerical-methods}{https://github.com/TiagoCAAmorim/numerical-methods}.

% \subsection{Método da Bissecção}
% \lstinputlisting[language=C, linerange={1-229}]{./02_newton_raphson.c}

% \subsection{Método de Newton-Raphson}
% \lstinputlisting[language=C, linerange={231-445}]{./02_newton_raphson.c}

% \subsection{Método da Mínima Curvatura}
% \lstinputlisting[language=C, linerange={448-958}]{./02_newton_raphson.c}

\end{document}
\endinput