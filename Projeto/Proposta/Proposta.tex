\documentclass[final,5p]{elsarticle}

% \documentclass[preprint,12pt]{elsarticle}

%% Use the option review to obtain double line spacing
%% \documentclass[authoryear,preprint,review,12pt]{elsarticle}

%% Use the options 1p,twocolumn; 3p; 3p,twocolumn; 5p; or 5p,twocolumn
%% for a journal layout:
% \documentclass[final,1p,times]{elsarticle}
%% \documentclass[final,1p,times,twocolumn]{elsarticle}
% \documentclass[final,3p,times]{elsarticle}
%% \documentclass[final,3p,times,twocolumn]{elsarticle}
% \documentclass[final,5p,times]{elsarticle}
%% \documentclass[final,5p,times,twocolumn]{elsarticle}
\usepackage[portuguese]{babel}

%% For including figures, graphicx.sty has been loaded in
%% elsarticle.cls. If you prefer to use the old commands
%% please give \usepackage{epsfig}

%% The amssymb package provides various useful mathematical symbols
\usepackage{amssymb}
\usepackage{amsmath}
\usepackage{multirow}
\usepackage{tabularx}
\usepackage{booktabs}
\usepackage{tablefootnote}

\usepackage{pgfplots}
\pgfplotsset{compat=1.18}
\usepgfplotslibrary{statistics}
\usepackage{pgfplotstable}

\usepackage{placeins}
\usepackage{hyperref}
\numberwithin{equation}{section}

\usepackage{algorithm}
\usepackage[noEnd=true, indLines=true]{algpseudocodex}
\algrenewcommand\algorithmicrequire{\textbf{Entrada:}}
\algrenewcommand\algorithmicwhile{\textbf{Enquanto}}
\algrenewcommand\algorithmicrepeat{\textbf{Repete}}
\algrenewcommand\algorithmicuntil{\textbf{Até}}
\algrenewcommand\algorithmicif{\textbf{Se}}
\algrenewcommand\algorithmicthen{\textbf{então}}
\algrenewcommand\algorithmicelse{\textbf{Caso contrário}}
\algrenewcommand\algorithmicensure{\textbf{Objetivo:}}
\algrenewcommand\algorithmicreturn{\textbf{Retorna:}}
\algrenewcommand\algorithmicdo{\textbf{faça}}
\algrenewcommand\algorithmicforall{\textbf{Para todos}}
\algnewcommand{\LineComment}[1]{\State \(\triangleright\) \textcolor{black!50}{\emph{#1}}}

\newcommand*{\squareb}{\textcolor{black}{\rule{0.5em}{0.5em}}}
\newcommand*{\squareg}{\textcolor{gray}{\rule{0.5em}{0.5em}}}

\graphicspath{ {./png/} }

% \usepackage[fleqn]{nccmath}
% \usepackage{multicol}


%=========== Gloabal Tikz settings
% \pgfplotsset{compat=newest}
% \usetikzlibrary{math}
% \pgfplotsset{
%     height = 10cm,
%     width = 10cm,
%     tick pos = left,
%     legend style={at={(0.98,0.30)}, anchor=east},
%     legend cell align=left,
%     }
%  \pgfkeys{
%     /pgf/number format/.cd,
%     fixed,
%     precision = 1,
%     set thousands separator = {}
% }

%% The amsthm package provides extended theorem environments
%% \usepackage{amsthm}

%% The lineno packages adds line numbers. Start line numbering with
%% \begin{linenumbers}, end it with \end{linenumbers}. Or switch it on
%% for the whole article with \linenumbers.
%% \usepackage{lineno}

\usepackage{listings}
\usepackage{xcolor}

\definecolor{codegreen}{rgb}{0,0.6,0}
\definecolor{codegray}{rgb}{0.5,0.5,0.5}
\definecolor{codepurple}{rgb}{0.58,0,0.82}
\definecolor{backcolour}{rgb}{0.98,0.98,0.98}

\lstdefinestyle{mystyle}{
    backgroundcolor=\color{backcolour},
    commentstyle=\color{codegreen},
    keywordstyle=\color{magenta},
    numberstyle=\tiny\color{codegray},
    stringstyle=\color{codepurple},
    basicstyle=\ttfamily\footnotesize,
    breakatwhitespace=false,
    breaklines=true,
    captionpos=b,
    keepspaces=true,
    numbers=left,
    numbersep=5pt,
    showspaces=false,
    showstringspaces=false,
    showtabs=false,
    tabsize=2
}

\lstset{style=mystyle}

% \journal{Nuclear Physics B}

\begin{document}

\begin{frontmatter}

%% Title, authors and addresses

%% use the tnoteref command within \title for footnotes;
%% use the tnotetext command for theassociated footnote;
%% use the fnref command within \author or \address for footnotes;
%% use the fntext command for theassociated footnote;
%% use the corref command within \author for corresponding author footnotes;
%% use the cortext command for theassociated footnote;
%% use the ead command for the email address,
%% and the form \ead[url] for the home page:
%% \title{Title\tnoteref{label1}}
%% \tnotetext[label1]{}
%% \author{Name\corref{cor1}\fnref{label2}}
%% \ead{email address}
%% \ead[url]{home page}
%% \fntext[label2]{}
%% \cortext[cor1]{}
%% \affiliation{organization={},
%%             addressline={},
%%             city={},
%%             postcode={},
%%             state={},
%%             country={}}
%% \fntext[label3]{}

\title{Modelos de Fluxo Substitutos com Redes Convolucionais Recorrentes\tnoteref{label_title}}
\tnotetext[label_title]{Proposta de projeto como parte dos requisitos da disciplina IA048: Aprendizado de Máquina.}

%% use optional labels to link authors explicitly to addresses:
%% \author[label1,label2]{}
%% \affiliation[label1]{organization={},
%%             addressline={},
%%             city={},
%%             postcode={},
%%             state={},
%%             country={}}
%%
%% \affiliation[label2]{organization={},
%%             addressline={},
%%             city={},
%%             postcode={},
%%             state={},
%%             country={}}

\author[label1]{Tiago Corrêa de Araújo de Amorim (RA: 100675) \texttt{t100675@dac.unicamp.br}}
\affiliation[label1]{organization={Doutorando no Departamento de Engenharia de Petróleo da Faculdade de Engenharia Mecânica, UNICAMP},
            city={Campinas},
            state={SP},
            country={Brasil}}

\author[label2]{\\Taylon Luan Congio Martins (RA: 177379) \texttt{t177379@m.unicamp.br}}
\affiliation[label2]{organization={Aluno especial, UNICAMP},
            city={Campinas},
            state={SP},
            country={Brasil}}

% \begin{abstract}
%     xxxxxxx
% \end{abstract}

%%Graphical abstract
% \begin{graphicalabstract}
%\includegraphics{grabs}
% \end{graphicalabstract}

%%Research highlights
% \begin{highlights}
% \item Research highlight 1
% \item Research highlight 2
% \end{highlights}

\begin{keyword}
    Fluxo em Meio Poroso \sep Redes Neurais Recorrentes \sep Rede Neurais Convolucionais
%% keywords here, in the form: keyword \sep keyword

%% PACS codes here, in the form: \PACS code \sep code

%% MSC codes here, in the form: \MSC code \sep code
%% or \MSC[2008] code \sep code (2000 is the default)

\end{keyword}

\end{frontmatter}

%% main text
\section{Introdução}

    Na indústria do petróleo é comum o uso de simuladores de fluxo tridimensionais para realizar a previsão de produção de acumulações de hidrocarbonetos \cite{rosa2006engenharia,dake1983fundamentals}. Estas simulações são empregadas em diferentes análises, desde a avaliação de valor para aquisição de blocos exploratórios, passando pelo planejamento de novos campos e, entre outros, a otimização de operações ao longo da explotação.

    O nível de sofisticação das formulações empregadas e o tamanho dos modelos de fluxo dependem das caracteríticas dos fluidos originais, técnicas utilizadas na extração de hidrocarbonetos, o tamanho e a heterogeneidade da acumulação. Mesmo fazendo uso de recursos de HPC, as simulações de fluxo de campos gigantes podem tomar dias para serem concluídas \cite{10.2118/149132-MS}.

    Diferentes proposta já foram feitas para tentar substituir parte da simulação de fluxo utilizando técnicas \emph{tradicionais} \cite{amorim2012risk} ou baseadas em inteligência artificial \cite{ertekin2019artificial}. Entre as diferentes propostas feitas com o uso de redes neurais é mais comum o foco em realizar novas previsões a partir de séries temporais \cite{de2022data, davtyan2020oil, kim2021recurrent} ou de modelos de fluxo completos \cite{cirac2023deep}.

    Uma proposta de arquitetura para aproximar física complexa foca no uso de redes em grafo para simular um passo de tempo \cite{sanchez2020learning}. Nesta rede o objetivo é que se aprenda a resolver o comportamento de um célula (ou partícula), dada a influência das células vizinhas, em um único passo de tempo. A proposta deste projeto é de aplicar algumas das ideias apresentadas no problema de fluxo em meio poroso.

\section{Proposta}

    A técnica mais utilizada na simulação de fluxo em meios porosos é a de diferenças finitas \cite{computer2022cmg,schlumberger2009technical}, que é aplicada em uma malha \emph{aproximadamente} regular. O foco deste projeto será limitado a malhas regulares, i.e., para modelos bidimensionais os vizinhos da célula $(i,j)$ são sempre $(i-1,j)$, $(i+1,j)$, $(i,j-1)$ e $(i,j+1)$\footnote{No caso tridimensional são 6 vizinhos.}. Desta forma não é necessário o uso de redes em grafo. Cada célula pode ser entendida como o pixel de uma imagem, e cada propriedade como um canal desta imagem. Camadas convolucionais serão suficientes para buscar representar a influência das células vizinhas.

    Como este problema apresenta simetria entre a célula e suas vizinhas, será testado o uso de camadas convolucionais com filtros (\emph{kernels}) com pesos simétricos.

    A arquitetura da rede deve seguir de forma geral a proposta apresentada em \cite{sanchez2020learning}:

    \begin{enumerate}
        \item \textbf{Encoder}: Transforma o estado das células para o espaço latente (filtros 1x1).
        \item \textbf{Processamento}: Computa as interações de cada célula com os seus vizinhos. O processamento é feito em \textbf{n} recorrências (filtros 3x3 \emph{simétricos}).
        \item \textbf{Decoder}: Retorna o estado das células para o espaço regular (filtros 1x1).
    \end{enumerate}

    O treinamento será feito com resultados de simulações de um passo de tempo. Para tal serão gerados resultados intermediários de diferentes simulações, com um dia de diferença entre os resultados.

    Como a proposta inclui a construção de camadas convolucionais com caracteríticas peculiares, está planejado o uso do pacote Pytorch \cite{paszke2019pytorch}. A base de dados será construída com as saídas de simulações de fluxo baseadas no modelo Unisim IV \cite{botechia2023unisim}. Uma primeira avaliação será feita com modelos bidimensionais. A depender dos resultados o estudo poderá ser extendido para modelos tridimensionais.

    O que se espera é construir uma rede que aprenda a gerar o estado futuro de uma célula em função do seu estado atual, dos controles de poços e da influência das células vizinhas. Desta forma esta rede pode ser utilizada de modo recursivo para gerar previsões de produção para diferentes valores das propriedades das células e dos mecanismos de controle de poço.

%% \label{}

%% If you have bibdatabase file and want bibtex to generate the
%% bibitems, please use
%%

\bibliographystyle{elsarticle-num}
\bibliography{refs}

%% else use the following coding to input the bibitems directly in the
%% TeX file.

% \begin{thebibliography}{00}

%% \bibitem{label}
%% Text of bibliographic item

% \bibitem{}

% \end{thebibliography}


\end{document}
\endinput
