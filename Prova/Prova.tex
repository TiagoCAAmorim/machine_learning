\documentclass[final,3p]{elsarticle}

% \documentclass[preprint,12pt]{elsarticle}

%% Use the option review to obtain double line spacing
%% \documentclass[authoryear,preprint,review,12pt]{elsarticle}

%% Use the options 1p,twocolumn; 3p; 3p,twocolumn; 5p; or 5p,twocolumn
%% for a journal layout:
% \documentclass[final,1p,times]{elsarticle}
%% \documentclass[final,1p,times,twocolumn]{elsarticle}
% \documentclass[final,3p,times]{elsarticle}
%% \documentclass[final,3p,times,twocolumn]{elsarticle}
% \documentclass[final,5p,times]{elsarticle}
%% \documentclass[final,5p,times,twocolumn]{elsarticle}
\usepackage[portuguese]{babel}

%% For including figures, graphicx.sty has been loaded in
%% elsarticle.cls. If you prefer to use the old commands
%% please give \usepackage{epsfig}

%% The amssymb package provides various useful mathematical symbols
\usepackage{amssymb}
\usepackage{amsmath}
\usepackage{multirow}
\usepackage{tabularx}
\usepackage{booktabs}
\usepackage{tablefootnote}

\usepackage{pgfplots}
\pgfplotsset{compat=1.18}
\usepgfplotslibrary{statistics}
\usepackage{pgfplotstable}

\usepackage{placeins}
\usepackage{hyperref}
\numberwithin{equation}{section}

\usepackage{algorithm}
\usepackage[noEnd=true, indLines=true]{algpseudocodex}
\algrenewcommand\algorithmicrequire{\textbf{Entrada:}}
\algrenewcommand\algorithmicwhile{\textbf{Enquanto}}
\algrenewcommand\algorithmicrepeat{\textbf{Repete}}
\algrenewcommand\algorithmicuntil{\textbf{Até}}
\algrenewcommand\algorithmicif{\textbf{Se}}
\algrenewcommand\algorithmicthen{\textbf{então}}
\algrenewcommand\algorithmicelse{\textbf{Caso contrário}}
\algrenewcommand\algorithmicensure{\textbf{Objetivo:}}
\algrenewcommand\algorithmicreturn{\textbf{Retorna:}}
\algrenewcommand\algorithmicdo{\textbf{faça}}
\algrenewcommand\algorithmicforall{\textbf{Para todos}}
\algnewcommand{\LineComment}[1]{\State \(\triangleright\) \textcolor{black!50}{\emph{#1}}}

\newcommand*{\squareb}{\textcolor{black}{\rule{0.5em}{0.5em}}}
\newcommand*{\squareg}{\textcolor{gray}{\rule{0.5em}{0.5em}}}

\graphicspath{ {./png/} }

% \usepackage[fleqn]{nccmath}
% \usepackage{multicol}


%=========== Gloabal Tikz settings
% \pgfplotsset{compat=newest}
% \usetikzlibrary{math}
% \pgfplotsset{
%     height = 10cm,
%     width = 10cm,
%     tick pos = left,
%     legend style={at={(0.98,0.30)}, anchor=east},
%     legend cell align=left,
%     }
%  \pgfkeys{
%     /pgf/number format/.cd,
%     fixed,
%     precision = 1,
%     set thousands separator = {}
% }

%% The amsthm package provides extended theorem environments
%% \usepackage{amsthm}

%% The lineno packages adds line numbers. Start line numbering with
%% \begin{linenumbers}, end it with \end{linenumbers}. Or switch it on
%% for the whole article with \linenumbers.
%% \usepackage{lineno}

\usepackage{listings}
\usepackage{xcolor}

\definecolor{codegreen}{rgb}{0,0.6,0}
\definecolor{codegray}{rgb}{0.5,0.5,0.5}
\definecolor{codepurple}{rgb}{0.58,0,0.82}
\definecolor{backcolour}{rgb}{0.98,0.98,0.98}

\lstdefinestyle{mystyle}{
    backgroundcolor=\color{backcolour},
    commentstyle=\color{codegreen},
    keywordstyle=\color{magenta},
    numberstyle=\tiny\color{codegray},
    stringstyle=\color{codepurple},
    basicstyle=\ttfamily\footnotesize,
    breakatwhitespace=false,
    breaklines=true,
    captionpos=b,
    keepspaces=true,
    numbers=left,
    numbersep=5pt,
    showspaces=false,
    showstringspaces=false,
    showtabs=false,
    tabsize=2
}

\lstset{style=mystyle}

% \journal{Nuclear Physics B}

\begin{document}

\begin{frontmatter}

%% Title, authors and addresses

%% use the tnoteref command within \title for footnotes;
%% use the tnotetext command for theassociated footnote;
%% use the fnref command within \author or \address for footnotes;
%% use the fntext command for theassociated footnote;
%% use the corref command within \author for corresponding author footnotes;
%% use the cortext command for theassociated footnote;
%% use the ead command for the email address,
%% and the form \ead[url] for the home page:
%% \title{Title\tnoteref{label1}}
%% \tnotetext[label1]{}
%% \author{Name\corref{cor1}\fnref{label2}}
%% \ead{email address}
%% \ead[url]{home page}
%% \fntext[label2]{}
%% \cortext[cor1]{}
%% \affiliation{organization={},
%%             addressline={},
%%             city={},
%%             postcode={},
%%             state={},
%%             country={}}
%% \fntext[label3]{}

\title{IA048 - Aprendizado de Máquina: Prova 28/05/2024}

%% use optional labels to link authors explicitly to addresses:
%% \author[label1,label2]{}
%% \affiliation[label1]{organization={},
%%             addressline={},
%%             city={},
%%             postcode={},
%%             state={},
%%             country={}}
%%
%% \affiliation[label2]{organization={},
%%             addressline={},
%%             city={},
%%             postcode={},
%%             state={},
%%             country={}}

\author{Tiago Corrêa de Araújo de Amorim (RA: 100675)}
% \affiliation[label1]{organization={Doutorando no Departamento de Engenharia de Petróleo da Faculdade de Engenharia Mecânica, UNICAMP},
            % city={Campinas},
            % state={SP},
            % country={Brasil}}

% \author[label2]{\\Taylon Luan Congio Martins (RA: 177379) \texttt{t177379@m.unicamp.br}}
% \affiliation[label2]{organization={Aluno especial, UNICAMP},
%             city={Campinas},
%             state={SP},
%             country={Brasil}}

% \begin{abstract}
%     xxxxxxx
% \end{abstract}

%%Graphical abstract
% \begin{graphicalabstract}
%\includegraphics{grabs}
% \end{graphicalabstract}

%%Research highlights
% \begin{highlights}
% \item Research highlight 1
% \item Research highlight 2
% \end{highlights}

% \begin{keyword}
%     Fluxo em Meio Poroso \sep Redes Neurais Recorrentes \sep Rede Neurais Convolucionais
% %% keywords here, in the form: keyword \sep keyword

% %% PACS codes here, in the form: \PACS code \sep code

% %% MSC codes here, in the form: \MSC code \sep code
% %% or \MSC[2008] code \sep code (2000 is the default)

% \end{keyword}

\end{frontmatter}

%% main text
\section{Questão 1}

    As redes neurais generativas adversárias (GANs, generative adversarial networks) trouxeram uma abordagem inovadora para a área de aprendizado de máquina.

    \begin{enumerate}[(a)]
        \item (0,8) Explique por que as redes geradora e discriminadora são consideradas adversárias.
        \item (0,8) Tendo em vista os papeis desempenhados por cada rede, discorra sobre a função custo proposta por Goodfellow et al. (2014)\footnotemark[01] para o treinamento da GAN. Complementando essa análise, comente também sobre como a rede geradora consegue aprender as características típicas dos dados reais.
    \end{enumerate}

    \footnotetext[01]{Goodfellow, I. J., Pouget-Abadie, J., Mirza, M., Xu, B., Warde-Farley, D., Ozair, S., Courville, A., Bengio, Y., “Generative Adversarial Networks”, Proceedings of the 27th International Conference on Neural Information Processing Systems (NIPS), pp.2672-2680, 2014.}

\section{Questão 2}

    (1,2) Explique os conceitos de mapa de características (feature map), neurônio, campo receptivo e compartilhamento de pesos no contexto de uma camada convolucional de uma CNN (convolutional neural network).

\section{Questão 3}

Tendo em mente a teoria subjacente às máquinas de vetores-suporte (SVMs, do inglês support-vector machines):

    \begin{enumerate}[(a)]
        \item (0,5) Defina margem de classificação no contexto de um problema linearmente separável e explique por que a maximização da margem é, intuitivamente, uma abordagem segura de projeto.
        \item (0,5) Por que a formulação matemática relacionada à obtenção do classificador de máxima margem possui similaridades com as ideias de regularização (e.g. Tikhonov / ridge regression) vistas no curso?
    \end{enumerate}

\section{Questão 4}

    (1,2) Numa importante conferência da área de inteligência computacional, foi lançada uma competição no âmbito de um problema de classificação de imagens. Duas jovens pesquisadoras decidiram, então, empregar uma CNN já consolidada na literatura (e.g. uma ResNet) para resolver a tarefa. Partindo de uma inicialização aleatória para os pesos da rede escolhida, e utilizando apenas os dados disponíveis na competição, elas, ao final do treinamento, observaram um nível de acurácia bastante adequado. No entanto, quando exposta a novos padrões de entrada (conjunto de teste), esta rede não atingiu um bom desempenho, tendo ficado muito abaixo das expectativas iniciais de projeto. Recomende (com argumentos bem fundamentados) duas estratégias para amenizar este problema.

\section{Questão 5}

    (1,4) Considere o problema de identificação de tumores cerebrais a partir de um conjunto de 1500 imagens de ressonância magnética (MRI) do crânio, de dimensão 200 × 200, em tons de cinza (veja um exemplo na Figura 1).
    As possíveis classes do problema são: (0) glioma; (1) meningioma; (2) pituitário; (3) saudável (normal). Descreva detalhadamente como uma rede MLP (multilayer perceptron), com uma única camada intermediária, poderia ser aplicada a este problema, indicando aspectos referentes à arquitetura da rede, bem como à metodologia como um todo (e.g. tratamento dos dados, treinamento do modelo etc.).

\section{Questão 6}

    Ao realizar um processo de PCA sobre uma base de dados, um pesquisador obteve os seguintes autovalores para a matriz de autocorrelação:

    \begin{equation}
        \mathbf{\lambda{}} =
        \begin{bmatrix}
            0,2 \\
            2,3 \\
            1,5 \\
            3 \\
            0,05 \\
            0,15 \\
            0,5 \\
            0,3
        \end{bmatrix}
    \end{equation}

    \begin{enumerate}[(a)]
        \item (0,3) Qual é a dimensão do espaço original dos dados?
        \item (0,7) Caso se busque uma preservação de ao menos 90\% do conteúdo energético dos dados, qual é o menor número possível de componentes principais empregadas? Justifique.
    \end{enumerate}


\section{Questão 7}

    Considere que uma CNN tenha sido aplicada à classificação de imagens de retina em três classes: normal, glaucoma e catarata. Na etapa de teste, a seguinte matriz de confusão foi obtida:

    \begin{table}[h]
        \centering
        \begin{tabular}{l c c c}
            \toprule
             & \multicolumn{3}{c}{\textbf{Classe estimada}} \\
            %  & \textbf{Classe} & \textbf{estimada} & \\
            \textbf{Classe real} & \textbf{Normal} & \textbf{Glaucoma} & \textbf{Catarata} \\
            \midrule
            Normal & 410 & 15 & 25 \\
            Glaucoma & 10 & 100 & 40 \\
            Catarata & 55 & 25 & 100 \\
            \bottomrule
        \end{tabular}
    \end{table}

    \begin{enumerate}[(a)]
        \item (0,4) Considerando a classe \textbf{Glaucoma} como positiva, obtenha as quantidades de verdadeiros positivos (TP), falsos positivos (FP), verdadeiros negativos (TN) e falsos negativos (FN).
        \item (0,6) Determine o valor da acurácia balanceada atingida pela CNN, mostrando explicitamente os valores das métricas intermediárias necessárias para o cálculo.
    \end{enumerate}

\section{Questão 8}

    \begin{enumerate}[(a)]
        \item (1,0) Explique como funciona o mecanismo de auto-atenção de um transformer.
        \item (0,6) De que maneira um transformer consegue aproveitar possíveis interdependências de curto e longo prazo em uma sequência de entrada apesar de não ter recorrência?
    \end{enumerate}



%% \label{}

%% If you have bibdatabase file and want bibtex to generate the
%% bibitems, please use
%%

% \bibliographystyle{elsarticle-num}
% \bibliography{refs}

%% else use the following coding to input the bibitems directly in the
%% TeX file.

% \begin{thebibliography}{00}

%% \bibitem{label}
%% Text of bibliographic item

% \bibitem{}

% \end{thebibliography}


\end{document}
\endinput
